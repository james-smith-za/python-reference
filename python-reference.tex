\documentclass{book}

\usepackage[cm]{fullpage}

\usepackage{listings}
\usepackage{color}
\lstset{%
backgroundcolor=\color{white},   % choose the background color; you must add \usepackage{color} or \usepackage{xcolor}
basicstyle=\footnotesize,        % the size of the fonts that are used for the code
breakatwhitespace=false,         % sets if automatic breaks should only happen at whitespace
breaklines=true,                 % sets automatic line breaking
captionpos=b,                    % sets the caption-position to bottom
commentstyle=\color{green},      % comment style
escapeinside={\%*}{*},          % if you want to add LaTeX within your code
extendedchars=true,              % lets you use non-ASCII characters; for 8-bits encodings only, does not work with UTF-8
frame=single,                      % adds a frame around the code
keepspaces=true,                 % keeps spaces in text, useful for keeping indentation of code (possibly needs columns=flexible)
keywordstyle=\color{blue},       % keyword style
identifierstyle=\color{red},
language=Python,                 % the language of the code
numbers=left,                    % where to put the line-numbers; possible values are (none, left, right)
numbersep=5pt,                   % how far the line-numbers are from the code
numberstyle=\tiny,               % the style that is used for the line-numbers
rulecolor=\color{black},         % if not set, the frame-color may be changed on line-breaks within not-black text (e.g. comments (green here))
showspaces=false,                % show spaces everywhere adding particular underscores; it overrides 'showstringspaces'
showstringspaces=false,          % underline spaces within strings only
showtabs=false,                  % show tabs within strings adding particular underscores
stepnumber=1,                    % the step between two line-numbers. If it's 1, each line will be numbered
stringstyle=\color{blue},        % string literal style
tabsize=2,                       % sets default tabsize to 2 spaces
title=\lstname                   %chktex 1; show the filename of files included with \lstinputlisting; also try caption instead of title
}%chktex 1

%Command for insertion of inline code snippets.
\definecolor{codegray}{gray}{0.9}
\newcommand{\code}[1]{\colorbox{codegray}{\texttt{#1}}}

\usepackage{url}

\begin{document}

\frontmatter

\title{Python reference for things I use}
\author{James Smith}
\date{\today}
\maketitle

\tableofcontents

\chapter{Introduction}

\mainmatter% 
\chapter{Strings and numbers}

There is lots of occasion to work with strings!

\section{Stripping non-numeric characters from a string}

If you're importing strings from a text file, but someone has mixed other things which aren't numbers in with the strings.
\lstinputlisting{src/non-numeric-strip.py} %chktex 11
This uses Python's list filtering. This is a very simple thing which doesn't require importing any additional modules such as \code{re}. This will only work for positive integers (TODO: verify this). If you've got negative numbers and points / commas, these are not digits, so won't get in.%chktex 13

Reference:
\url{http://stackoverflow.com/questions/17336943/removing-non-numeric-characters-from-a-string}



\chapter{optparser}

\lstinputlisting[language=Python]{src/optparser-ex.py} %chktex 11



\appendix
\chapter{if needed}

\backmatter%
\chapter{Last note}

\end{document}

